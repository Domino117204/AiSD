\documentclass{article}

% Pakiety wymagane do formatowania i grafiki
\usepackage{geometry}  % Ustawienie marginesów
\usepackage{graphicx}  % Wymagany do wstawiania obrazków
\usepackage{xcolor}    % Kolorowy tekst
\usepackage[T1]{fontenc}
\usepackage[utf8]{inputenc}
\usepackage{polski}
\usepackage{lmodern} 
\usepackage{float}
\usepackage{pgfplots}
\usepackage{titlesec}
\usepackage{multicol}
\usepackage{listings}

\usepackage{fancyhdr} % Pakiet do nagłówków i stopek
\pagestyle{fancy} % Ustawienie niestandardowego stylu


\fancyhead[L]{\includegraphics[width=10cm]{Image/PP-PUT-WORD.png}} % obraz w lewym górnym rogu


\fancyfoot[C]{\thepage}

\usepackage{tcolorbox}
\definecolor{terminalColor}{RGB}{40, 42, 54}
\definecolor{Button1}{RGB}{237, 106, 94}
\definecolor{Button2}{RGB}{245, 191, 79}
\definecolor{Button3}{RGB}{98, 188, 119}



\pgfplotsset{
	width=10cm, height=6cm, ymin=0, xmin=4, xmin=2047, xmax=65536, grid=both, %wymiary osi
	xticklabel style={rotate=45, anchor=near xticklabel},  %liczby na osi X pod kątem 45*
	x label style={at={(axis description cs:0.5,-0.025)}}, %legenda osi X wyśrodkowana, lekko obniżona
}

\renewcommand{\headrulewidth}{0pt} % Usunięcie linii w nagłówku
\renewcommand{\footrulewidth}{0pt} % Cienka linia stopki

% Marginesy strony
\geometry{
	a4paper,
	left=20mm,
	right=20mm,
	top=30mm,
	bottom=25mm,
	headsep=20mm, % Odstęp między nagłówkiem a tekstem
}


% Formatowanie sekcji
\titleformat{\section}{\LARGE\bfseries\centering}{}{0em}{}

\begin{document}
	
	\lstset{
		backgroundcolor=\color{black},  % Tło czarne
		basicstyle=\ttfamily\color{white},  % Kolor tekstu biały
		keywordstyle=\color{cyan},  % Kolor słów kluczowych na niebiesko
		commentstyle=\color{gray},  % Kolor komentarzy na szaro
		stringstyle=\color{green},  % Kolor ciągów znaków na zielono
		showstringspaces=false,  % Nie pokazuj spacji w ciągach
		tabsize=3,  % Rozmiar tabulacji
		breaklines=true,  % Łamanie długich linii
		frame=none,  % Bez ramki
		xleftmargin=0cm,  % Margines po lewej stronie
		xrightmargin=0cm,  % Margines po prawej stronie
		aboveskip=0pt,  % Odstęp przed
		belowskip=0pt,  % Odstęp po
		columns=fullflexible,  % Kolumny elastyczne
		linewidth=\linewidth  % Szerokość linii na pełną szerokość
	}
	
	% Strona tytułowa
	\thispagestyle{empty} % Brak nagłówków/stopek na pierwszej stronie
	
	\begin{center}
		\includegraphics[width=6cm]{Image/PP-PUT-LOGO.png}
		\vspace{1cm}
		
		{\Huge\bfseries Politechnika Poznańska}
		
		\vspace{1cm}
		
		{\large Informatyka rok I semestr 2} \\[0.3cm]
		L10, Piątek 0:00 - 0:00
		
		\vspace{1.5cm}
		
		{\LARGE\bfseries Algorytmy i Struktury Danych} \\[0.3cm]
		\textbf{Prowadzący:} Dominik Piotr Witczak
		
		\vspace{2cm}
		
		% Poprawione formatowanie
		{\LARGE\bfseries Sprawozdanie nr 1} 
		
		\vspace{1cm}
		
		{\Large\bfseries Algorytmy Sortowania}
		
		\vspace{3cm}
		
		\begin{flushright}
			\textbf{Autor:} \\[0.2cm]
			Dominik Fischer 164176 \\
			Oliwer Miller
		\end{flushright}
		
		\vspace{2cm}
		Rok akademicki 2024/2025
	\end{center}
	

	
	\newpage
	
	\titleformat{\section}{\Huge\bfseries}{\thesection}{2em}{}
	
	\section*{\textcolor{blue}{Wprowadzenie}}
	
	Tutaj piszemy wprowadzenie
	
	\titleformat{\section}{\Large\bfseries}{\thesection}{2em}{}
	\section*{\textcolor{blue}{Selection Sort}}
	\begin{multicols}{2}
		\noindent Tutaj opis algorytmu
		
		\noindent 
		\begin{tcolorbox}[colback=black,colframe=gray!50!,arc=3mm,boxrule=0pt,left=0pt,right=0pt,width=\linewidth]
			\textcolor{white}{\textbf{\textsf{Terminal}}}\\
			
			\begin{lstlisting}[language=Python]
def selection_sort(data):
	n=len(data)
	for j in range(n-1):
		min = j
		for i in range(j+1, n):
			if data[i] < data[min]:
				min = i
		data[j], data[min] = data[min], data[j]
	return data
			\end{lstlisting}
			
		\end{tcolorbox}
	\end{multicols}
	
	\section*{\textcolor{blue}{Insertion Sort}}
	\begin{multicols}{2}
		\noindent Tutaj opis algorytmu
		
		\noindent 
		\begin{tcolorbox}[colback=black,colframe=gray!50!,arc=3mm,boxrule=0pt,left=0pt,right=0pt,width=\linewidth]
			\textcolor{white}{\textbf{\textsf{Terminal}}}\\
			
			\begin{lstlisting}[language=Python]
def insertion_sort (data):
	for i in range(1, len(data)):
		key = data[i]
		j = i - 1
		while j >= 0 and data[j] > key:
			data[j + 1] = data[j]
			j -= 1
		data[j + 1] = key
	return data
			\end{lstlisting}
			
		\end{tcolorbox}
	\end{multicols}
	
	\newpage
	
	\section*{\textcolor{blue}{shell Sort With Sadgewick Gaps}}
	\begin{multicols}{2}
		\noindent Tutaj opis algorytmu
		
		\noindent 
		\begin{tcolorbox}[colback=black,colframe=gray!50!,arc=3mm,boxrule=0pt,left=0pt,right=0pt,width=\linewidth]
			\textcolor{white}{\textbf{\textsf{Terminal}}}\\
			
			\begin{lstlisting}[language=Python]
def sedgewick_gaps(n):
	gaps = []
	k = 0
	while True:
		if k % 2 == 0:
			gap = 9 * (2 ** k) - 9 * (2 ** (k // 2)) + 1
		else:
			gap = 4 ** k + 3 * 2 ** (k - 1) + 1

		if gap >= n:
			break
		gaps.append(gap)
		k += 1

	return gaps[::-1]

def shell_sort(data):
	n = len(data)
	gaps = sedgewick_gaps(n)

	for gap in gaps:
		for i in range(gap, n):
			temp = data[i]
			j = i
			while j >= gap and data[j - gap] > temp:
				data[j] = data[j - gap]
				j -= gap
			data[j] = temp
	return data
			\end{lstlisting}
			
		\end{tcolorbox}
	\end{multicols}
	
	\newpage
	
	\section*{\textcolor{blue}{Heap Sort}}
	\begin{multicols}{2}
		\noindent Tutaj opis algorytmu
		
		\noindent 
		\begin{tcolorbox}[colback=black,colframe=gray!50!,arc=3mm,boxrule=0pt,left=0pt,right=0pt,width=\linewidth]
			\textcolor{white}{\textbf{\textsf{Terminal}}}\\
			
			\begin{lstlisting}[language=Python]
def heap(data,n,i):
	largest=i
	left = 2*i+1
	right = 2*i+2

	if left<n and data[left]>data[largest]:
		largest = left
	if right<n and data[right]>data[largest]:
		largest=right
	if largest !=i:
		data[i], data[largest]=data[largest],data[i]
		heap(data, n, largest)

def heap_sort(data):
	n = len(data)

	for i in range(n // 2 - 1, -1, -1):  
		heap(data, n, i)

	for i in range(n - 1, 0, -1):
		data[i], data[0] = data[0], data[i]
		heap(data, i, 0)

	return data
			\end{lstlisting}
			
		\end{tcolorbox}
	\end{multicols}
	
	\newpage
	
	\section*{\textcolor{blue}{Quick Sort Left Pivot}}
	\begin{multicols}{2}
		\noindent Tutaj opis algorytmu
		
		\noindent 
		\begin{tcolorbox}[colback=black,colframe=gray!50!,arc=3mm,boxrule=0pt,left=0pt,right=0pt,width=\linewidth]
			\textcolor{white}{\textbf{\textsf{Terminal}}}\\
			
			\begin{lstlisting}[language=Python]
def partition(A, p, r):
	pivot = A[p]
	i = p+1
	j = r

	while True:
		while i <= j and A[i] <= pivot:
			i += 1
		while i <= j and A[j] > pivot:
			j -= 1
		if i <= j:
			A[i], A[j] = A[j], A[i]
		else:
			break

	A[p], A[j] = A[j], A[p]
	return j

def quick_sort_left_pivot(A, p, r):
	if p < r:
		q = partition(A, p, r)
		quick_sort_left_pivot(A, p, q-1)
		quick_sort_left_pivot(A, q+1, r)
	return A
			\end{lstlisting}
			
		\end{tcolorbox}
	\end{multicols}
	
	\newpage
	
	\section*{\textcolor{blue}{Quick Sort Random Pivot}}
	\begin{multicols}{2}
		\noindent Tutaj opis algorytmu
		
		\noindent 
		\begin{tcolorbox}[colback=black,colframe=gray!50!,arc=3mm,boxrule=0pt,left=0pt,right=0pt,width=\linewidth]
			\textcolor{white}{\textbf{\textsf{Terminal}}}\\
			
			\begin{lstlisting}[language=Python]
def quick_sort_random_pivot(data):
	if len(data) <= 1:
		return data

	pivot_index = random.randint(0, len(data) - 1)
	pivot = data[pivot_index]

	left = []
	middle = []
	right = []

	for i, x in enumerate(data):
		if i == pivot_index:
			middle.append(x)
		elif x < pivot:
			left.append(x)
		elif x > pivot:
			right.append(x)

	return quick_sort_random_pivot(left) + middle + quick_sort_random_pivot(right)

			\end{lstlisting}
			
		\end{tcolorbox}
	\end{multicols}
	
	\newpage
	
	%\section*{WYKRESY}
	%Wykresy, obrazy itd, warto opakowywać w `figure`, dzięki czemu można dodać Caption\Opis do figury, jak i `label` dzięki któremu później można odwoływać się do figur np. ``
	
	
	\section*{\textcolor{blue}{Porównanie czasów wykonania}}
	\begin{figure}[H]
		\centering
		%Tworzę "pojemnik" na wykres `pgfplots` tak żeby automatycznie wyskalował mi wygenerowany wykres na szerokość strony. 
		\noindent\resizebox{\textwidth}{!}{
			\begin{tikzpicture}
				\begin{axis}[%
					name=plotA, anchor=left of south west,
					title={Złożoność Obliczeniowa Algorytmu Heap Sort}, 
					xlabel={Rozmiar instancji}, ylabel={Czas(ms)}, legend pos=north west,
					xmode = log, log basis x={2},
					every axis plot post/.style={very thick},
					/tikz/plot label/.style={black, anchor=west}
					]
					\addplot[red, dashed, smooth] table[x=InputSize,y=Time,meta=Algorithm,col sep=comma] {results/heap_sort_a_shaped_array.csv};
					\addplot[green, dashed, smooth] table[x=InputSize,y=Time,meta=Algorithm,col sep=comma] {results/heap_sort_constant_array.csv};
					\addplot[blue, dashed, smooth] table[x=InputSize,y=Time,meta=Algorithm,col sep=comma] {results/heap_sort_decreasing_array.csv};
					\addplot[yellow, dashed, smooth] table[x=InputSize,y=Time,meta=Algorithm,col sep=comma] {results/heap_sort_increasing_array.csv};
					\addplot[black, dashed, smooth] table[x=InputSize,y=Time,meta=Algorithm,col sep=comma] {results/heap_sort_random_array.csv};
					\legend{A shaped, Constant, Decresing, Incresing, Random}
				\end{axis}
				
				
				%Prawy górny
				\begin{axis}[%
					title={Złożoność Obliczeniowa Algorytmu Insertion Sort}, 
					name=plotB, at=(plotA.right of south east), anchor=left of south west,
					xlabel={Rozmiar instancji}, ylabel={Czas(ms)}, legend pos=north west,
					xmode = log, log basis x={2},
					every axis plot post/.style={very thick},
					/tikz/plot label/.style={black, anchor=west}
					]
					\addplot[red, dashed, smooth] table[x=InputSize,y=Time,meta=Algorithm,col sep=comma] {results/insertion_sort_a_shaped_array.csv};
					\addplot[green, dashed, smooth] table[x=InputSize,y=Time,meta=Algorithm,col sep=comma] {results/insertion_sort_constant_array.csv};
					\addplot[blue, dashed, smooth] table[x=InputSize,y=Time,meta=Algorithm,col sep=comma] {results/insertion_sort_decreasing_array.csv};
					\addplot[yellow, dashed, smooth] table[x=InputSize,y=Time,meta=Algorithm,col sep=comma] {results/insertion_sort_increasing_array.csv};
					\addplot[black, dashed, smooth] table[x=InputSize,y=Time,meta=Algorithm,col sep=comma] {results/insertion_sort_random_array.csv};
					\legend{A shaped, Constant, Decresing, Incresing, Random}
				\end{axis}
				
				
				%Trzeci wykres
				\begin{axis}[%
					title={Złożoność Obliczeniowa Algorytmu Selection Sort}, 
					name=plotD, at=(plotB.below south west), anchor=above north west,
					xlabel={Rozmiar instancji}, ylabel={Czas(ms)}, legend pos=north west,
					xmode = log, log basis x={2},
					every axis plot post/.style={very thick},
					/tikz/plot label/.style={black, anchor=west}
					]
					\addplot[red, dashed, smooth] table[x=InputSize,y=Time,meta=Algorithm,col sep=comma] {results/selection_sort_a_shaped_array.csv};
					\addplot[green, dashed, smooth] table[x=InputSize,y=Time,meta=Algorithm,col sep=comma] {results/selection_sort_constant_array.csv};
					\addplot[blue, dashed, smooth] table[x=InputSize,y=Time,meta=Algorithm,col sep=comma] {results/selection_sort_decreasing_array.csv};
					\addplot[yellow, dashed, smooth] table[x=InputSize,y=Time,meta=Algorithm,col sep=comma] {results/selection_sort_increasing_array.csv};
					\addplot[black, dashed, smooth] table[x=InputSize,y=Time,meta=Algorithm,col sep=comma] {results/selection_sort_random_array.csv};
					\legend{A shaped, Constant, Decresing, Incresing, Random}
				\end{axis}
				%Czwarty wykres
				\begin{axis}[%
					title={Złożoność Obliczeniowa Algorytmu Shell Sort}, 
					name=plotC, at=(plotD.left of south west), anchor=right of south east,
					xlabel={Rozmiar instancji}, ylabel={Czas(ms)}, legend pos=north west,
					xmode = log, log basis x={2},
					every axis plot post/.style={very thick},
					/tikz/plot label/.style={black, anchor=west}
					]
					\addplot[red, dashed, smooth] table[x=InputSize,y=Time,meta=Algorithm,col sep=comma] {results/shell_sort_a_shaped_array.csv};
					\addplot[green, dashed, smooth] table[x=InputSize,y=Time,meta=Algorithm,col sep=comma] {results/shell_sort_constant_array.csv};
					\addplot[blue, dashed, smooth] table[x=InputSize,y=Time,meta=Algorithm,col sep=comma] {results/shell_sort_decreasing_array.csv};
					\addplot[yellow, dashed, smooth] table[x=InputSize,y=Time,meta=Algorithm,col sep=comma] {results/shell_sort_increasing_array.csv};
					\addplot[black, dashed, smooth] table[x=InputSize,y=Time,meta=Algorithm,col sep=comma] {results/shell_sort_random_array.csv};
					\legend{A shaped, Constant, Decresing, Incresing, Random}
				\end{axis}
				
				
				
				
				%Piaty wykres
				\begin{axis}[%
					title={Złożoność Obliczeniowa Algorytmu Quick Sort Left Pivot}, 
					name=plotF, at=(plotD.below south west), anchor=above north west,
					xlabel={Rozmiar instancji}, ylabel={Czas(ms)}, legend pos=north west,
					xmode = log, log basis x={2},
					every axis plot post/.style={very thick},
					/tikz/plot label/.style={black, anchor=west}
					]
					\addplot[red, dashed, smooth] table[x=InputSize,y=Time,meta=Algorithm,col sep=comma] {results/quick_sort_left_pivot_a_shaped_array.csv};
					\addplot[green, dashed, smooth] table[x=InputSize,y=Time,meta=Algorithm,col sep=comma] {results/quick_sort_left_pivot_a_shaped_array.csv};
					\addplot[blue, dashed, smooth] table[x=InputSize,y=Time,meta=Algorithm,col sep=comma] {results/quick_sort_left_pivot_decreasing_array.csv};
					\addplot[yellow, dashed, smooth] table[x=InputSize,y=Time,meta=Algorithm,col sep=comma] {results/quick_sort_left_pivot_increasing_array.csv};
					\addplot[black, dashed, smooth] table[x=InputSize,y=Time,meta=Algorithm,col sep=comma] {results/quick_sort_left_pivot_random_array.csv};
					\legend{A shaped, Constant, Decresing, Incresing, Random}
				\end{axis}
				%Szosty wykres
				\begin{axis}[%
					title={Złożoność Obliczeniowa Algorytmu Quick Sort Random Pivot}, 
					name=plotE, at=(plotF.left of south west), anchor=right of south east,
					xlabel={Rozmiar instancji}, ylabel={Czas(ms)}, legend pos=north west,
					xmode = log, log basis x={2},
					every axis plot post/.style={very thick},
					/tikz/plot label/.style={black, anchor=west}
					]
					\addplot[red, dashed, smooth] table[x=InputSize,y=Time,meta=Algorithm,col sep=comma] {results/quick_sort_random_pivot_a_shaped_array.csv};
					\addplot[green, dashed, smooth] table[x=InputSize,y=Time,meta=Algorithm,col sep=comma] {results/quick_sort_random_pivot_constant_array.csv};
					\addplot[blue, dashed, smooth] table[x=InputSize,y=Time,meta=Algorithm,col sep=comma] {results/quick_sort_random_pivot_decreasing_array.csv};
					\addplot[yellow, dashed, smooth] table[x=InputSize,y=Time,meta=Algorithm,col sep=comma] {results/quick_sort_random_pivot_increasing_array.csv};
					\addplot[black, dashed, smooth] table[x=InputSize,y=Time,meta=Algorithm,col sep=comma] {results/quick_sort_random_pivot_random_array.csv};
					\legend{A shaped, Constant, Decresing, Incresing, Random}
				\end{axis}
				
			\end{tikzpicture}
		}
	\end{figure}
	
	
	
	
	
	\section*{\textcolor{blue}{Porównanie czasów wykonania poszczególnych algorytmów względem danych}}
	\begin{figure}[H]
		\centering
		\label{fig:enter-label}
		%Tworzę "pojemnik" na wykres `pgfplots` tak żeby automatycznie wyskalował mi wygenerowany wykres na szerokość strony. 
		\noindent\resizebox{\textwidth}{!}{
			\begin{tikzpicture}
				\begin{axis}[%
					name=plotA, anchor=left of south west,
					title={A shaped}, 
					xlabel={Rozmiar instancji}, ylabel={Czas(ms)}, legend pos=north west,
					xmode = log, log basis x={2},
					every axis plot post/.style={very thick},
					/tikz/plot label/.style={black, anchor=west}
					]
					\addplot[red, dashed, smooth] table[x=InputSize,y=Time,meta=Algorithm,col sep=comma] {results/heap_sort_a_shaped_array.csv};
					\addplot[green, dashed, smooth] table[x=InputSize,y=Time,meta=Algorithm,col sep=comma] {results/insertion_sort_a_shaped_array.csv};
					\addplot[blue, dashed, smooth] table[x=InputSize,y=Time,meta=Algorithm,col sep=comma] {results/shell_sort_a_shaped_array.csv};
					\addplot[yellow, dashed, smooth] table[x=InputSize,y=Time,meta=Algorithm,col sep=comma] {results/selection_sort_a_shaped_array.csv};
					\addplot[black, dashed, smooth] table[x=InputSize,y=Time,meta=Algorithm,col sep=comma] {results/quick_sort_random_pivot_a_shaped_array.csv};
					\addplot[gray, dashed, smooth] table[x=InputSize,y=Time,meta=Algorithm,col sep=comma] {results/quick_sort_left_pivot_a_shaped_array.csv};
					\legend{Heap Sort, Insertion Sort, Shell Sort, Selection Sort, Quick Sort Random, Quick Sort Left}
				\end{axis}
				
				
				%Prawy górny
				\begin{axis}[%
					title={Constant}, 
					name=plotB, at=(plotA.right of south east), anchor=left of south west,
					xlabel={Rozmiar instancji}, ylabel={Czas(ms)}, legend pos=north west,
					xmode = log, log basis x={2},
					every axis plot post/.style={very thick},
					/tikz/plot label/.style={black, anchor=west}
					]
					\addplot[red, dashed, smooth] table[x=InputSize,y=Time,meta=Algorithm,col sep=comma] {results/heap_sort_constant_array.csv};
					\addplot[green, dashed, smooth] table[x=InputSize,y=Time,meta=Algorithm,col sep=comma] {results/insertion_sort_constant_array.csv};
					\addplot[blue, dashed, smooth] table[x=InputSize,y=Time,meta=Algorithm,col sep=comma] {results/shell_sort_constant_array.csv};
					\addplot[yellow, dashed, smooth] table[x=InputSize,y=Time,meta=Algorithm,col sep=comma] {results/selection_sort_constant_array.csv};
					\addplot[black, dashed, smooth] table[x=InputSize,y=Time,meta=Algorithm,col sep=comma] {results/quick_sort_random_pivot_constant_array.csv};
					\addplot[gray, dashed, smooth] table[x=InputSize,y=Time,meta=Algorithm,col sep=comma] {results/quick_sort_left_pivot_constant_array.csv};
					\legend{Heap Sort, Insertion Sort, Shell Sort, Selection Sort, Quick Sort Random, Quick Sort Left}
				\end{axis}
				
				
				%Trzeci wykres
				\begin{axis}[%
					title={Decresing}, 
					name=plotD, at=(plotB.below south west), anchor=above north west,
					xlabel={Rozmiar instancji}, ylabel={Czas(ms)}, legend pos=north west,
					xmode = log, log basis x={2},
					every axis plot post/.style={very thick},
					/tikz/plot label/.style={black, anchor=west}
					]
					\addplot[red, dashed, smooth] table[x=InputSize,y=Time,meta=Algorithm,col sep=comma] {results/heap_sort_decreasing_array.csv};
					\addplot[green, dashed, smooth] table[x=InputSize,y=Time,meta=Algorithm,col sep=comma] {results/insertion_sort_decreasing_array.csv};
					\addplot[blue, dashed, smooth] table[x=InputSize,y=Time,meta=Algorithm,col sep=comma] {results/shell_sort_decreasing_array.csv};
					\addplot[yellow, dashed, smooth] table[x=InputSize,y=Time,meta=Algorithm,col sep=comma] {results/selection_sort_decreasing_array.csv};
					\addplot[black, dashed, smooth] table[x=InputSize,y=Time,meta=Algorithm,col sep=comma] {results/quick_sort_random_pivot_decreasing_array.csv};
					\addplot[gray, dashed, smooth] table[x=InputSize,y=Time,meta=Algorithm,col sep=comma] {results/quick_sort_left_pivot_decreasing_array.csv};
					\legend{Heap Sort, Insertion Sort, Shell Sort, Selection Sort, Quick Sort Random, Quick Sort Left}
				\end{axis}
				%Czwarty wykres
				\begin{axis}[%
					title={Increasing}, 
					name=plotC, at=(plotD.left of south west), anchor=right of south east,
					xlabel={Rozmiar instancji}, ylabel={Czas(ms)}, legend pos=north west,
					xmode = log, log basis x={2},
					every axis plot post/.style={very thick},
					/tikz/plot label/.style={black, anchor=west}
					]
					\addplot[red, dashed, smooth] table[x=InputSize,y=Time,meta=Algorithm,col sep=comma] {results/heap_sort_increasing_array.csv};
					\addplot[green, dashed, smooth] table[x=InputSize,y=Time,meta=Algorithm,col sep=comma] {results/insertion_sort_increasing_array.csv};
					\addplot[blue, dashed, smooth] table[x=InputSize,y=Time,meta=Algorithm,col sep=comma] {results/shell_sort_increasing_array.csv};
					\addplot[yellow, dashed, smooth] table[x=InputSize,y=Time,meta=Algorithm,col sep=comma] {results/selection_sort_increasing_array.csv};
					\addplot[black, dashed, smooth] table[x=InputSize,y=Time,meta=Algorithm,col sep=comma] {results/quick_sort_random_pivot_increasing_array.csv};
					\addplot[gray, dashed, smooth] table[x=InputSize,y=Time,meta=Algorithm,col sep=comma] {results/quick_sort_left_pivot_increasing_array.csv};
					\legend{Heap Sort, Insertion Sort, Shell Sort, Selection Sort, Quick Sort Random, Quick Sort Left}
				\end{axis}
				
				
				
				
				%Piaty wykres
				\begin{axis}[%
					title={Random}, 
					name=plotF, at=(plotC.below south), anchor=above north west,
					xlabel={Rozmiar instancji}, ylabel={Czas(ms)}, legend pos=north west,
					xmode = log, log basis x={2},
					every axis plot post/.style={very thick},
					/tikz/plot label/.style={black, anchor=west}
					]
					\addplot[red, dashed, smooth] table[x=InputSize,y=Time,meta=Algorithm,col sep=comma] {results/heap_sort_random_array.csv};
					\addplot[green, dashed, smooth] table[x=InputSize,y=Time,meta=Algorithm,col sep=comma] {results/insertion_sort_random_array.csv};
					\addplot[blue, dashed, smooth] table[x=InputSize,y=Time,meta=Algorithm,col sep=comma] {results/shell_sort_random_array.csv};
					\addplot[yellow, dashed, smooth] table[x=InputSize,y=Time,meta=Algorithm,col sep=comma] {results/selection_sort_random_array.csv};
					\addplot[black, dashed, smooth] table[x=InputSize,y=Time,meta=Algorithm,col sep=comma] {results/quick_sort_random_pivot_random_array.csv};
					\addplot[gray, dashed, smooth] table[x=InputSize,y=Time,meta=Algorithm,col sep=comma] {results/quick_sort_left_pivot_random_array.csv};
					\legend{Heap Sort, Insertion Sort, Shell Sort, Selection Sort, Quick Sort Random, Quick Sort Left}
				\end{axis}
				
				
			\end{tikzpicture}
		}
	\end{figure}
	\section*{\textcolor{blue}{Wnioski}}
	Tutaj dajemy wnioski
	
	
\end{document}